% Introduction

Content-free grammar has enabled several "almost-real" publication generators for several areas such as mathematics, computer science, and high-energy physics. Content-free grammar is a set of rules that are independent from the words (or phrases) they apply to. By defining a dictionary containing field-related keywords, and a set of rules similar to languages used in research publications, it can generate any number of publications. Famous examples of such generator is SCIgen, an automatic CS paper generator; snarXiv, a random high-energy theory paper generator similar to arXiv; the Real Theorem Generator for mathematical theorems, and Philosophy of the Day, a philosophy sentence generator.

Both SCIgen and snarXiv have proved that human are not good at distinguishing computer-generated papers from real papers. One of the SCIgen-generated paper was accepted by WMSCI 2005, an influential multi-discipline conference in computer science, and snarXiv reports that classification between real paper titles and machine-generated titles by human is merely 59\%, slightly higher than random.

Therefore, we aim to build a classification mechanism to distinguish machine-generated contents to real, human-produced contents. There are very few publications on this topic. A SCIgen paper detection method was described in this paper by using inter-textual distance (based on bag-of-words) and knn classification method. The author claims it should "hardly" misclassify, but no performance metric is given.

% unfinished
We focus on paper title and abstract because snarXiv can only generate these two parts. Thus far, we have crawled data from arXiv.org, providing positive data points; acquired the source for the snarXiv abstract generator to generate negative data points; developed a framework for informed feature selection; and have begun evaluating our feature transform methods.